\documentclass[12pt]{article}
\usepackage[english]{babel}
\usepackage{longtable}
\usepackage[top=1in, bottom=1.25in, left=1.25in, right=1.25in,includefoot,heightrounded]{geometry}
\usepackage{indentfirst}
\usepackage[utf8]{inputenc}
\usepackage{amsmath,amssymb}
\usepackage{graphicx,tikz}
\usepackage{hyperref}
\usepackage[colorinlistoftodos]{todonotes}
\usepackage[document]{ragged2e}
\usepackage{fancyhdr}
\usepackage{enumerate}
\usepackage{listings}
\usepackage{color}
%\usepackage{flowchart}
\usepackage{hyperref}
\usetikzlibrary{arrows}


\usetikzlibrary{shapes.geometric, arrows}
\tikzstyle{startstop} = [rectangle, rounded corners, minimum width=3cm, minimum height=1cm,text centered, draw=black, fill=red!30]
\tikzstyle{decision} = [diamond, minimum width=4cm, minimum height=0.5cm, text centered, draw=black, fill=green!30]
\tikzstyle{process} = [rectangle, minimum width=3cm, minimum height=1cm, text centered, draw=black, fill=orange!30]
\tikzstyle{arrow} = [thick,->,>=stealth]
\tikzstyle{io} = [trapezium, trapezium left angle=70, trapezium right angle=110, minimum width=2cm, text width=4cm, minimum height=1cm, text centered, draw=black, fill=blue!30]

\pagestyle{fancy}
\fancyhf{}
\rhead{Dongfang Wang}
\lhead{Concept-Instance Project}
\cfoot{\thepage}
\renewcommand{\headrulewidth}{0.4pt}
\renewcommand{\baselinestretch}{1.2}

\renewcommand{\labelitemi}{$\circ$}

\definecolor{MyDarkGreen}{rgb}{0.0,0.4,0.0} % This is the color used for comments
\lstloadlanguages{java} % Load C syntax for listings, for a list of other languages supported see: ftp://ftp.tex.ac.uk/tex-archive/macros/latex/contrib/listings/listings.pdf
\lstset{language=java, 
        %frame=single,  Single frame around code
        basicstyle=\small\ttfamily, % Use small true type font
        keywordstyle=[1]\color{blue}\bf, % C functions bold and blue
        keywordstyle=[2]\color{purple}, % C function arguments purple
        keywordstyle=[3]\color{blue}\underbar, % Custom functions underlined and blue
        identifierstyle=, % Nothing special about identifiers                                         
        commentstyle=\usefont{T1}{pcr}{m}{sl}\color{MyDarkGreen}\small, % Comments small dark green courier font
        stringstyle=\color{purple}, % Strings are purple
        showstringspaces=false, % Don't put marks in string spaces
        tabsize=5, % 5 spaces per tab
        %
        % Put standard Perl functions not included in the default language here
        morekeywords={rand},
        %
        % Put Perl function parameters here
        morekeywords=[2]{on, off, interp},
        %
        % Put user defined functions here
        morekeywords=[3]{test},
       	%
        morecomment=[l][\color{blue}]{...}, % Line continuation (...) like blue comment
        numbers=left, % Line numbers on left
        firstnumber=1, % Line numbers start with line 1
        numberstyle=\tiny\color{blue}, % Line numbers are blue and small
        stepnumber=1 % Line numbers go in steps of 5
}
\lstset{literate=%
{à}{{\`a}}1
{é}{{\'e}}1
{è}{{\`e}}1
{í}{{\'i}}1
{ò}{{\`o}}1
{ó}{{\'o}}1
{ú}{{\'u}}1
{ç}{{\c{c}}}1
}
%\lstset{extendedchars=\true}
\lstset{inputencoding=ansinew}
\lstset{breaklines=true} %, postbreak=\raisebox{0ex}[0ex][0ex]{\ensuremath{\color{red}\hookrightarrow\space}}}
\begin{document}

\section*{Summary of this week}

Recently, I've been tired of tempting different gene selection methods,
different clustering methods, different parameters, etc., to obtain appropriate
consensus clustering results. Finally something interesting was found, and I'm trying
to summarize them as follows.

\subsection*{PAC score}
It's not trival to select a plausible estimated number of clusters for any
clustering methods. There exist some works
offering some heuristic methods to do this. {\bf \color{blue}PAC}(Proportion of
Ambiguously Clustered pairs), especially the rounded PAC score, is a relatively
reasonal standard for consensus clustering methods. 

\vspace{5pt}
Each element of the consensus matrix represents the proportion of
a sample pair being clustered into the same cluster during the re-sampling
procedure, with 0 meaning this pair never appears in the same cluster, and 1
meaning this pair always appear in the same cluster.

\vspace{5pt}
So we set two thresholds,
\[ u_1=0.1,\qquad u_2=0.9 \]
and PAC is the proportion of sample pairs locating in the region $[u_1,u_2]$. So
samller PAC means more stable clustering results.

\vspace{5pt}
Finally, we use the $K$ with the samllest PAC score. For example, in the
following graph, $K=5$ will be selected.

\begin{figure}[!h]
  \centering
  \includegraphics[width=.5\textwidth,height=185pt]{./fig/consensus.pdf}
 \end{figure}

 \subsection*{Implementation}

 Next, I implement consensus clutering on the BREAST cancer datasets. Some
 packages I use are listed as follows.
 \begin{itemize}
 \item {\bf \color{blue}'ConsensusClustePlus'}. This is a Bioconductor-based package, which
   implements all regular inner clustering methods for consensus clustering.
   Something strange exists in this package, although. It offers
   interface for users-defined clusterint methods, which, however, are only
   allowed to accept distance matrix and number of clusters as input. This, of
   course, can fit most clustering methods, but not all, such as the basic
   kmeans methods. So why do we modify the basic kmeans, just to look for
   troubles? Unfortunately, the kmeans embedded inside the package is the
   default parameters of {\bf R}'s kmeans function, which sets $iter.max=10,\
   nstart=1$, using only one random start, and 10 iterations. This has great
   possibilies not to converge.
 \item {\bf\color{blue}'curatedBreastData'}. This package collects 34 public
   available breast cancer datasets. This package contains also a function
   {\color{blue}filterAndImputeSamples } to impute NAs in the gene expression matrix.
 \end{itemize}

During the debug process, I find the following parameters may have some
substantial effects:
\begin{itemize}
\item The number of genes selected and the selection strategy
\item The inner clustering algorithms
\item The implementation of clustering algorithms
\item ...
\end{itemize}
Before I elaborate on some experiment results, I must admit by adjusting the
above parameters alone or all, I didn't obtain the expected results. It may seem
weird to me,that by accident, I checked the detailed implementations of {\bf
  \color{blue}COINCIDE} package, and they use $pItem=.9$ instead of the $0.8$ I
used. This parameter controls the proportion of re-sampled samples, i.e. there
will be $pItem \times N$ samples sampled from the original datasets. I don't
know why this is, but I obseved in my experiments.

Now, for others, I should summarize some heuristic strategies. The main inner
clustering methods include kmeans and hierarchical clutering. The former is
good, but we should use 100 iterations. The latter should be used carefully. The
'complete' linkage method is should not be used, which usually clusters most
patients in one cluster and leaves extreme small number of patients.

I will only use kmeans in the following discussions. I summarize results in the table.
\begin{longtable}{c|l|cccc}
%\centeringi
%\label{my-label}
%\begin{tabular}{c|l|cccc}
\hline
 \bf\color{blue} NO.&\bf\color{blue}NAME&\bf\color{blue}Samples&\bf\color{blue}Gene&\bf\color{blue} Optimal K&\bf\color{blue} PAC  \\
\hline
  1 & GPL1223\_all(1379) & 60-2 &0.05& 4 & 0.08 \\
    & & & 0.1 & 5 & 0.006 \\
  & & & 0.5 & 5 & 0.17 \\ \hline
  2 & GPL96\_all(2034) & 286-5 & 0.05 & 2 & 0.05  \\
                    & & & 0.1 & 7 & 0.11 \\
  & & & 0.5 & 7 & 0.09 \\ \hline
  3 & GPL3558\_all(4913) & 50-2 & 0.05 & 7 & 0.11 \\
                    & & & 0.1 & 6& 0.17\\
  & & & 0.5 & 2 & 0.08\\ \hline
  4 & GPL3883\_all(6577) & 88-4 & 0.05 & 4 & 0.11\\
                    & & & 0.1 & 4 & 0.13\\
  & & & 0.5 & 4 & 0.14\\ \hline
  5 & GPL\_5049\_all(9893) & 155-30 & 0.01 & 10 & 0.27\\
                     & & & 0.1 & 10 & 0.26 \\
  & & & 0.5 & 4 & 0.19 \\ \hline
  6 & GPL5186\_all(12071) & 46-2 & 0.05 & -&- \\
                      & & & 0.1 & 10 & 0.14 \\
  & & & 0.5 & 3 & 0.12 \\ \hline
  7 & GPL96\_all(12093) & 136-10 & 0.05 & 2 & 0.107\\
                       & & & 0.1 &2 &0.107  \\
  & & & 0.5 & 2 & 0.05  \\ \hline
  8 & GPL570\_all(16391) & 48-2 & 0.05 & & \\
                       & & & 0.1 & 10 & 0.16 \\
  & & & 0.5 & 2 & 0 \\ \hline
  9 & GPL570\_aLL(16446) & 114-6 & 0.05 & & 0 \\
                        & & & 0.1 &2 & 0 \\
  & & & 0.5 & 2 & 0.02 \\ \hline
  10 & GPL\_JBI(17705) & 103-2 & 0.05 & 6 & 0.16 \\
                       & & & 0.1 &10 &0.18  \\
  & & & 0.5 & 2&0.17  \\ \hline
 
  11 & GPL\_MDACC(17705) & 195-11 & 0.05 & 2 & 0.07 \\
                        & & & 0.1 &3 &0.08  \\
  & & & 0.5 &3 & 0.09 \\ \hline
  12 & GPL570\_all(18728) & 21 &- & - & -\\
  \hline
  13 & GPL570\_all(19615) & 115-5 & 0.05 &2 & 0.02 \\
                        & & & 0.1 &2 &0.06  \\
  & & & 0.5 &2 & 0.09 \\ \hline
  14 & GPL570\_all(19697) & 24 & - & -&- \\
  \hline
  15 & GPL96\_all(20181) & 53-2 & 0.05 & 4 & 0.19\\
                        & & & 0.1 & 10 & 0.17 \\
  & & & 0.5 &10 & 0.16 \\ \hline
  16 & GPL96\_all(20194) & 261-11 & 0.05 & 7 & 0.10\\
                        & & & 0.1 &2 &0.07  \\
  & & & 0.5 & 6 & 0.08 \\ \hline
  17 & GPL6480\_all(21974) & 32 & - & -&- \\
  \hline
  18 & GPL1390\_all(21997) & 35 & - & -&- \\
  \hline
  19 & GPL5325\_all(21997) & 28 & - & -&- \\
  \hline
  20 & GPL7504\_all(21997) & 31 &- & -&- \\
  \hline
  21 & GPL1708\_all(22226) & 128-9 & 0.05 & 4 & 0.12 \\
                        & & & 0.1 &3 &0.11  \\
  & & & 0.5 & 7 & 0.12  \\ \hline
  22 & GPL4133\_all(22226) & 20 & - & - &- \\
  \hline
  23 & GPL5325\_all(22358) & 122-8 & 0.05 & 4 & 0.08 \\
                        & & & 0.1 & 4& 0.07 \\
  & & & 0.5 & 5 & 0.12 \\ \hline
  24 & GPL5325\_all(23428) & 16 & - & -&- \\
  \hline
  25 & GPL96\_MDACC(25055) & 221-12 & 0.05 & 2 & 0.06 \\
                        & & & 0.1 & 3 & 0.06 \\
  & & & 0.5 & 3 & 0.03  \\ \hline
  26 & GPL96\_MDACC(25055) & 6 & - & -&- \\
  \hline
  27 & GPL96\_LBJ(2506 5) & 17 & - &  -&- \\
  \hline
  28 & GPL96\_MDACC(25065) & 71-2 & 0.05 &2 & 0.03 \\
                        & & & 0.1 & 2& 0.03 \\
  & & & 0.5 & 2 & 0 \\ \hline
  29 & GPL96\_MDACC\_MDA(25065) & 15-1 & -& -&- \\
  \hline
  30 & GPL96\_PERU(25065) & 25-1 & - & -&- \\
  \hline
  31 & GPL96\_Spain(25065) & 16-2 & - &  -&- \\
  \hline
  32 & GPL96\_US0(25065) & 54-1 & 0.05 & 2 & 0.04 \\
                        & & & 0.1 &2 & 0.04 \\
  & & & 0.5 & 2 & 0.07 \\ \hline
  33 & GPL570\_all(32646) & 115-7 & 0.05 & 3 & 0.02 \\
                        & & & 0.1 & 2&0.07  \\
  & & & 0.5 & 10 & 0.16 \\ \hline
  34 & GPL570\_all(33658) & 11 & - & - & - \\
  \hline
%\end{tabular}
\end{longtable}

Basically, we can say there's no significant effects of the number of genes
selected if PAC is not very large.

{\bf Note:} I check the consensusClusetrPlus.R and find that they adopt a
replace-free re-sampling strategy. So stange...

All my results are stored on the server:
/data/home/dwang/project/conceptInstance/ConsensusClusterPlusImplementation

\end{document}